



\section{Introduction}

\begin{frame}[label=introduction]
\frametitle{Introduction}
\framesubtitle{RBE Lecture Template Overview}

This template provides a class for creating course lectures in \LaTeX.  

\end{frame}



\section{Container}

\begin{frame}[label=buildContainer]

Containers are provided with this template to support rapid development:

\begin{columns}
  \begin{column}{0.5\textwidth}
    Development container for VSCode
  \end{column}
  \begin{column}{0.5\textwidth}
    Standalone container for rapid builds
  \end{column}
\end{columns}


\end{frame}

\section{Features}



\begin{frame}[label=URLFRAME]
\frametitle{Embedded Links}


\begin{itemize}
    \item \href{https://www.wpi.edu}{WPI}
    \item \href{https://google.com}{Google}
    \item \href{http://www.weather.gov}{The NWS}
\end{itemize}


\end{frame}



\begin{frame}[label=MATHFRAME]
\frametitle{Equations}
\framesubtitle{Now it's time to do some math}

Embed equations in slides:

\begin{equation}
2 \cdot a^{2} \cdot x^{3} - b^{2} \cdot x^{2} + c \cdot x + 5 = 0
\end{equation}

Source: \cite{LaValle}


\end{frame}



\begin{frame}[fragile,label=CODEFRAME]
\frametitle{Source Listings}
\framesubtitle{A short coding example}

\begin{lstlisting}[language=C,basicstyle=\ttfamily]

#include<stdio.h>

int main(int argc, char** argv)
{
    printf("Hello world!\n");
    // BLEH
    return 0;
}

\end{lstlisting}

\end{frame}


\begin{frame}
    \frametitle{Figures with Sources}


    \begin{columns}
        \begin{column}{0.7\textwidth}

            A generic sourced figure, using \\citeFigure

        \end{column}
        \begin{column}{0.3\textwidth}

            % FIXME: test cite figure better (maybe add photo)
            \citeFigure{WPI_Inst_Prim_FulClr.png}{LaValle}

        \end{column}
    \end{columns}


\end{frame}



\begin{frame}[allowframebreaks]
\frametitle{Algorithms}

% FIXME: probably need to switch from algorithm to algorithmic to get frame breaks to work correctly

Embed algorithms (with frame breaks):

\begin{algorithm}[H]

\SetKwData{tree}{$\mathcal{T}$}

\SetKwData{Xstart}{$X_{start}$}
\SetKwData{Xsample}{$X_{sample}$}
\SetKwData{Xnearest}{$X_{nearest}$}
\SetKwData{Xnew}{$X_{new}$}
\SetKwData{Xgoal}{$X_{goal}$}
\SetKwData{Xrand}{$X_{rand}$}
\SetKwData{Xinit}{$X_{init}$}

\SetKwData{Sgoal}{$s_{goal}$}

\SetKwData{k}{$k$}
\SetKwData{kk}{$K$}


\SetKwData{statespace}{$X$}

\SetKwData{toss}{toss}
\SetKwData{heads}{heads}

\SetKwData{init}{init}

\SetKwData{extend}{EXTEND}
\SetKwData{randomstate}{RANDOM\_STATE()}
\SetKwData{cointoss}{COIN\_TOSS()}

\SetKwFunction{buildRRT}{BUILD\_RRT}
\SetKwFunction{biasedRandomState}{BIASED\_RANDOM\_STATE}
\SetKwProg{RRTalg}{}{}{}

\RRTalg{\buildRRT{\Xinit}}{




\tree.\init (\Xinit)

\For{\k $ = 1$ \KwTo \kk}{

    \Xrand $\leftarrow$ \alert{\biasedRandomState{}}

    \extend (\tree, \Xrand)

}

\KwRet{\tree}

}


\SetKwProg{biasedRRTfunc}{}{}{}

\biasedRRTfunc{\biasedRandomState{}}{

\toss $\leftarrow$ \cointoss

\If{\toss $=$ \heads}{
    \KwRet{\Sgoal}
}
\Else{
    \KwRet{\randomstate}

}

}


\caption{Goal-biased RRT algorithm}
\label{alg:goalBiasedRRT}
\end{algorithm}




\end{frame}

